\documentclass[12pt, a4paper]{article}
\usepackage[margin = 2cm]{geometry}
% Vertical text spacing
\parindent = 0cm \parskip = 0cm
% Section
\usepackage[compact]{titlesec} \titlespacing*{\section}{0pt}{2ex}{2ex}
\titleformat*{\section}{\normalfont\Large\bfseries\color[RGB]{0,0,192}}
\titleformat*{\subsection}{\normalfont\bfseries\color[RGB]{192,0,0}}
% Table spacing
\newcommand\TS{\rule{0pt}{2.6ex}}         % Top strut
\newcommand\BS{\rule[-0.9ex]{0pt}{0pt}}   % Bottom strut
\usepackage{array, multirow}
\newcolumntype{L}[1]{>{\raggedright\TS\BS\arraybackslash}m{#1}}
\newcolumntype{C}[1]{>{\centering  \TS\BS\arraybackslash}m{#1}}
\newcolumntype{R}[1]{>{\raggedleft \TS\BS\arraybackslash}m{#1}}
% Equations
\usepackage{amsmath, bm, bbold, tikz}

\title{\vspace{-6ex} CO395 Group 57 \vspace{-1ex}}
\author{Dongxiao Huang, Zheng Xun Phang, Yufeng Zhang}
\date{\vspace{-3ex}}

\begin{document}
\maketitle
\newcommand\ones{\bm{1}}

\section*{Q1}
\subsection*{Linear layers}
If each image in $X$ is reshaped into a row vector, then the forward pass can be written as
\[ z = XW + \ones b^T \]
For a linear layer, the partial derivatives are
\[ \frac{\partial z}{\partial X} = W \qquad \frac{\partial z}{\partial W} = X \qquad \frac{\partial z}{\partial b} = \ones \]
To compute derivatives of the loss function $L$, we apply the chain rule
\[ \frac{\partial L}{\partial X} = \frac{\partial L}{\partial z} \frac{\partial z}{\partial X} \qquad \frac{\partial L}{\partial W} = \frac{\partial L}{\partial z} \frac{\partial z}{\partial W} \qquad \frac{\partial L}{\partial b} = \frac{\partial L}{\partial z} \frac{\partial z}{\partial b} \]
Of course, images are not row vectors, so they must be reshaped appropriately.

\subsection*{ReLU activation}
The forward pass of ReLU is $\max(0, x)$.\par
\bigskip
Its derivative is 1 for $x \geq 0$ and 0 for $x < 0$. We apply the chain rule just like the linear layers.

\section*{Q2}
During training, the forward pass of dropout will multiply the output of some neurons by 0, so it's effectively removing neurons. We can set the proportion $p$ of neurons to dropout, and scale the outputs of the remaining neurons by $1/(1-p)$.\par
\bigskip
During testing, we restore all neurons.

\section*{Q3}
The softmax function $\sigma : \Re^C \rightarrow [0, 1]^C$ can represent a probability distribution over $C$ classes:
\[ \sigma(z_1, \dots, z_C) = \frac{1}{\exp(z_1) + \dots + \exp(z_C)}
   \begin{bmatrix} \exp(z_1) \\ \vdots \\ \exp(z_C) \end{bmatrix} :=
   \begin{bmatrix} \sigma_1  \\ \vdots \\ \sigma_C  \end{bmatrix} \]
We used the ``normalization trick'' described in the coursework manual for numerical stability.\par
\bigskip
Derivatives of the softmax function are
\[ \frac{\partial \sigma_i}{\partial z_j} = (\delta_{i,j} - \sigma_j) \, \sigma_i \qquad \forall \, i, j \in \{1, \dots, C\} \]
where $\delta_{i,j} = 0$ if $i \neq j$, otherwise $\delta_{i,j} = 1$.\par
\bigskip
The cross entropy loss of $n$ images is
\[ L = -\frac1n \sum_{i=1}^n \left[ y_{i,1} \log \sigma_{i,1} + \dots + y_{i,C} \log \sigma_{i,C} \right] \]
where $y_{i,k} = 1$ if image $i$ belongs to class $k$, otherwise $y_{i,k} = 0$. Similarly, $\sigma_{i,k}$ is the softmax probability that image $i$ belongs to class $k$.\par
\bigskip
Derivatives of the loss function for $k \in \{1, \dots, C\}$ are
\begin{align*}
    \frac{\partial L}{\partial z_k}
    &= -\frac1n \sum_{i=1}^n \left[ \frac{y_{i,1}}{\sigma_{i,1}} \, \frac{\partial \sigma_{i,1}}{\partial z_k} + \dots + \frac{y_{i,C}}{\sigma_{i,C}} \, \frac{\partial \sigma_{i,C}}{\partial z_k} \right] \\
    &= -\frac1n \sum_{i=1}^n \left[ y_{i,1} \, (\delta_{1,k} - \sigma_k) + \dots + y_{i,C} \, (\delta_{C,k} - \sigma_k) \right] \\
    &= -\frac1n \sum_{i=1}^n \left( y_{i,1} \, \delta_{1,k} + \dots + y_{i,C} \, \delta_{C,k} - \sigma_k \right) \qquad \qquad \text{since } y_{i,1} + \dots + y_{i,C} = 1
\end{align*}

\section*{Q4}
Architecture + parameters used to overfit a small subset of the CIFAR-10 data, plots of loss and accuracy of train and test set (2 points)

Architecture + parameters used to achieve at least 50\% accuracy on CIFAR-10 as well as plots of loss and accuracy of train and test set (2 points)

\section*{Q5}
Explanation of how the network was fine-tuned, what strategies were employed. Include plots of accuracy, loss and F1 if appropriate. (6 points for each step , 30 points in total)

Test the performance of the network trained with the optimal set of parameters on the test set and report the confusion matrix, classification rate and F1 measure per class. (5 points)

\section*{A1}
No free lunch theorem

\section*{A2}
No change is needed since our functions are written to accommodate an arbitrary number of classes.

\end{document}
