\documentclass[12pt, a4paper]{article}
\usepackage[margin = 2cm]{geometry}
\usepackage{caption} % package for captions
% Vertical text spacing
\parindent = 0cm \parskip = 0cm
% Section
\usepackage[compact]{titlesec} 
\titlespacing*{\section}{0pt}{2ex}{2ex}
\titleformat*{\section}{\normalfont\Large\bfseries\color[RGB]{0,0,192}}
\titleformat*{\subsection}{\normalfont\bfseries\color[RGB]{192,0,0}}
% Table spacing
\newcommand\TS{\rule{0pt}{2.6ex}}         % Top strut
\newcommand\BS{\rule[-0.9ex]{0pt}{0pt}}   % Bottom strut
\usepackage{array, multirow}
\newcolumntype{L}[1]{>{\raggedright\TS\BS\arraybackslash}m{#1}}
\newcolumntype{C}[1]{>{\centering  \TS\BS\arraybackslash}m{#1}}
\newcolumntype{R}[1]{>{\raggedleft \TS\BS\arraybackslash}m{#1}}
% Equations
\usepackage{amsmath, bm, bbold, tikz}

\title{\vspace{-6ex} CO395 Group 57 \vspace{-1ex}}
\author{Dongxiao Huang, Zheng Xun Phang, Yufeng Zhang}
\date{\vspace{-3ex}}

\begin{document}
\maketitle
\newcommand\ones{\bm{1}}

\section*{Question 1}

\subsection*{Linear layers}
If each image in $X$ is reshaped into a row vector, then the forward pass can be written as
\[ z = XW + \ones b^T \]
For a linear layer, the partial derivatives are
\[ \frac{\partial z}{\partial X} = W \qquad \frac{\partial z}{\partial W} = X \qquad \frac{\partial z}{\partial b} = \ones \]
To compute derivatives of the loss function $L$, we apply the chain rule
\[ \frac{\partial L}{\partial X} = \frac{\partial L}{\partial z} \frac{\partial z}{\partial X} \qquad \frac{\partial L}{\partial W} = \frac{\partial L}{\partial z} \frac{\partial z}{\partial W} \qquad \frac{\partial L}{\partial b} = \frac{\partial L}{\partial z} \frac{\partial z}{\partial b} \]
Of course, images are not row vectors, so they must be reshaped appropriately.

\subsection*{ReLU activation}
The forward pass of ReLU is $\max(0, x)$.\par
\bigskip
Its derivative is 1 for $x > 0$ and 0 for $x < 0$. At $x = 0$, its derivative does not exist, but its subderivative  lies between 0 and 1, so we simply set it to 0. We apply the chain rule just like the linear layers.

\section*{Question 2}

\subsection*{Training Phase}
During training, the forward pass of inverted dropout will multiply the output of some neurons by 0, so it's effectively removing neurons. We set the proportion $p$ of neurons to dropout, and scale the outputs of the remaining neurons by $1/(1-p)$.

\subsection* {Testing Phase}
During testing, we use all neurons in both forward and backward pass, as if they were never dropped out.

\section*{Question 3}

The softmax function $\sigma : \Re^C \rightarrow [0, 1]^C$ can represent a probability distribution over $C$ classes:
\[ \sigma(z_1, \dots, z_C) = \frac{1}{\exp(z_1) + \dots + \exp(z_C)}
   \begin{bmatrix} \exp(z_1) \\ \vdots \\ \exp(z_C) \end{bmatrix} :=
   \begin{bmatrix} \sigma_1  \\ \vdots \\ \sigma_C  \end{bmatrix} \]
We used the ``normalization trick'' described in the coursework manual for numerical stability.\par
\bigskip
Derivatives of the softmax function are
\[ \frac{\partial \sigma_i}{\partial z_j} = (\delta_{i,j} - \sigma_j) \, \sigma_i \qquad \forall \, i, j \in \{1, \dots, C\} \]
where $\delta_{i,j} = 0$ if $i \neq j$, otherwise $\delta_{i,j} = 1$.\par
\bigskip
The cross entropy loss of $n$ images is
\[ L = -\frac1n \sum_{i=1}^n \left[ y_{i,1} \log \sigma_{i,1} + \dots + y_{i,C} \log \sigma_{i,C} \right] \]
where $y_{i,k} = 1$ if image $i$ belongs to class $k$, otherwise $y_{i,k} = 0$. Similarly, $\sigma_{i,k}$ is the softmax probability that image $i$ belongs to class $k$.\par
\bigskip
Derivatives of the loss function for $k \in \{1, \dots, C\}$ are
\begin{align*}
    \frac{\partial L}{\partial z_k}
    &= -\frac1n \sum_{i=1}^n \left[ \frac{y_{i,1}}{\sigma_{i,1}} \, \frac{\partial \sigma_{i,1}}{\partial z_k} + \dots + \frac{y_{i,C}}{\sigma_{i,C}} \, \frac{\partial \sigma_{i,C}}{\partial z_k} \right] \\
    &= -\frac1n \sum_{i=1}^n \left[ y_{i,1} \, (\delta_{1,k} - \sigma_{i,k}) + \dots + y_{i,C} \, (\delta_{C,k} - \sigma_{i,k}) \right] \\
    &= \frac1n \sum_{i=1}^n \left( \sigma_{i,k} - y_{i,1} \, \delta_{1,k} - \dots - y_{i,C} \, \delta_{C,k} \right) \qquad \qquad \text{since } y_{i,1} + \dots + y_{i,C} = 1
\end{align*}
Note that $y_{i,1} \, \delta_{1,k} + \dots + y_{i,C} \, \delta_{C,k}$ is 1 when $k$ is such that $y_{i,k} = 1$, and it is 0 for other values of $k$. So computing the derivative involves going through each row of
\[ \begin{bmatrix}
   \sigma_{1,1} & \dotsm & \sigma_{1,C} \\
   & \vdots & \\
   \sigma_{n,1} & \dotsm & \sigma_{n,C}
   \end{bmatrix} \]
and subtracting 1 from the appropriate column.

\section*{Question 4}

The shapes of our input and output neurons are $32 \times 32 \times 3$ and 10 respectively.\par
\bigskip
Our activation function is ReLU and optimization algorithm is Stochastic Gradient Descent.\par
\bigskip
As a sanity check, we overfitted 50 images in the CIFAR-10 dataset using the architecture and hyperparameters in column A of the table below.\par
\bigskip
To achieve at least 50\% accuracy on the CIFAR-10 dataset, we used the architecture and hyperparameters in column B.
\begin{center}
\begin{tabular} { C{4cm} | C{2.5cm} C{3cm} }
    & A & B \\ \hline
    Hidden Neurons      & [50]  & [50, 8] \\
    Dropout				& None	& None  \\
    Regularization      & None  & L2 with $\lambda = 0$ \\
    Epochs              &   20  &   20  \\
    Batch Size          &  100  &  100  \\
    Learning Rate       & 0.001 & 0.005 \\
    Learning Rate Decay & 0.95  & 0.95
\end{tabular}
\end{center}
\bigskip

Plots of loss function and classification accuracy are provided below.

\begin{figure} [h!]
    \begin{minipage}{0.5\textwidth}
    \centering
    \includegraphics[width=\linewidth]{Figure_1.png}
    \caption{Overfitting} \label{Fig:Data1}
    \end{minipage}
    \begin {minipage}{0.5\textwidth}
    \centering
    \includegraphics[width=\linewidth]{train_fcnet.png}
    \caption{50\% accuracy} \label{Fig:Data2}
    \end{minipage}
\end{figure}

\section*{Question 5}

\subsection*{Step 1: Architecture Design}
A reasonable architecture to start with is: 2 hidden layers with 512 and 128 neurons respectively, since 512 and 128 are between our input size ($48 \times 48 = 2304$ pixels) and output size (7 labels). This is a common rule of thumb; there is no consensus on what a good architecture is anyway.\par
\bigskip
We initialized the neural net's weights from a standard Normal distribution independently.\par
\bigskip
We used the default momentum rate of 0.9 and the default learning rate update schedule, which decays the learning rate by 0.95.\par
\bigskip
For a start, we used the default ``stopping criterion'' in \texttt{solver.py} which is simply the number of iterations. We tried early stopping but it actually led to lower classification accuracy.\par
\bigskip

\subsection*{Step 2: Learning Rate}
In Figure \ref{learn_rate}, we plotted both training and validation classification accuracy (after 9000 iterations) for various learning rates, which range from $10^{-5}$ to $10^{-2}$ on a log scale.\par
\bigskip
Why 9000 iterations? There are about 270 iterations per epoch and we trained our neural net for 35 epochs, for a total of 9450 iterations.\par
\bigskip
The optimal learning rate is the one with highest validation accuracy i.e. lowest asymptotic error. This turns out to be around $5 \times 10^{-4}$ and it results in a training accuracy of 72\% (error of 28\%) and a validation accuracy of 42\% (error of 58\%).\par
\bigskip

\begin{figure} [h!]
    \begin{minipage}{\textwidth}
    \centering
    \includegraphics[width=\linewidth] {lr_compare.jpeg}
    \caption{Learning Rates} \label{learn_rate}
   \end{minipage} \hfill
\end{figure}

\subsection*{Step 3: Dropout}
We dropped out various proportions of neurons and plotted the training and test accuracy in Figure \ref{dropout}.\par
\bigskip

\begin{figure} [h!]
    \centering
    \includegraphics[width=\linewidth]{dropout_compare.jpeg}
    \caption{Dropout} \label{dropout}
\end{figure}

Dropout does not improve training and test accuracies. This suggests that our neural net might be underfitting, so we should probably increase the number of neurons.\par
\bigskip

\subsection*{Step 4: L2 Regularization}
We vary the L2 penalty rate $\lambda$ from 0 to 0.5 and plotted the training loss, training accuracy and validation accuracy in Figure \ref{L2}.\par
\bigskip
Just like dropout, L2 regularization does not improve validation accuracy i.e. it is maximized when $\lambda = 0$. This is consistent with our dropout finding that our neural net is underfitting.\par
\bigskip
Between dropout and regularization, we prefer the latter because it's more easily reproducible even without access to our code. Dropout has an element of arbitrariness.\par
\bigskip

\begin{figure} [h!]
    \centering
    \includegraphics[width=\linewidth] {l2_compare.jpg}
    \caption{L2 regularization} \label{L2}
\end{figure}

\subsection*{Step 5: Topology}
We have experimented with the following architectures:
\begin{itemize}
    \item (Wide) 1 layer with 1024 neurons
    \item (Deep) 5 layers with 512 neurons each
    \item (Narrowing) 3 layers with [512, 256, 128] neurons
    \item (Widening) 3 layers with [128, 256, 512] neurons
    \item (Fan in and out) 3 layers with [512, 128, 512] neurons
\end{itemize}
The last architecture is inspired by variational auto-encoders. We have also varied the number of neurons for the ``narrowing'' and ``widening'' architectures.\par
\bigskip

\subsection*{Final Architecture and Performance}
Our best neural net has [512, 512, 512] neurons and it was trained with these hyperparameters: number of epochs = 35, batch size = 100, learning rate = 0.005 and learning rate decay = 0.95\par
\bigskip
Confusion matrix based on test data:\par
\begin{center}
\begin{tabular} { C{2.7cm} | C{1.5cm} C{1.5cm} C{1.3cm} C{1.7cm} C{1.5cm} C{1.5cm} C{1.5cm} }
    \multirow{2}{*}{Actual Class} & \multicolumn{7}{c}{Predicted Class} \\
    & Anger & Disgust & Fear & Happiness & Sadness & Surprise & Neutral \\ \hline
    Anger     & 157 &  3 &  48 &  73 &  88 &  18 &  80 \\
    Disgust   &   6 & 26 &   4 &   9 &   1 &   2 &   8 \\
    Fear      &  46 &  2 & 166 &  59 & 101 &  42 &  80 \\
    Happiness &  62 &  2 &  35 & 568 &  95 &  32 & 101 \\
    Sadness   &  81 &  1 &  75 & 105 & 241 &  26 & 124 \\
    Surprise  &  22 &  0 &  73 &  24 &  28 & 275 &  29 \\
    Neutral   &  39 &  1 &  44 & 107 & 124 &  26 & 266
\end{tabular}
\end{center}
\bigskip
Note: It makes no sense to compute classification rate for each emotion separately. For example, if the actual emotion is Anger then (Fear, Surprise) counts as a true negative for Anger even though it is a misclassification! In other words, if an instance of Fear is misclassified as Surprise, it should not be taken as evidence that our neural nets are good at detecting Anger.\par
\bigskip
Summary statistics from the confusion matrix:
\begin{center}
\begin{table} [h!]
\begin{tabular} { C{2.7cm} | C{1.5cm} C{1.5cm} C{1.3cm} C{1.7cm} C{1.5cm} C{1.5cm} C{1.5cm} }
    & Anger & Disgust & Fear & Happiness & Sadness & Surprise & Neutral \\ \hline
    Precision   & 0.380 & 0.743 & 0.373 & 0.601 & 0.355 & 0.653 & 0.387 \\
    Recall      & 0.336 & 0.464 & 0.335 & 0.635 & 0.369 & 0.610 & 0.438 \\
    $F_1$ score & 0.357 & 0.571 & 0.353 & 0.617 & 0.362 & 0.631 & 0.411
\end{tabular}
\[ \text{Classification rate} = \frac{157 + 26 + \dotsm + 266}{3625} = 0.469 \]
\caption{Simple Neural Network} \label{simple}
\end{table}
\end{center}

\section*{Question A1}

The dataset of the previous coursework has only 1004 images, which is far too few for training a neural network as it will overfit. With more observations and appropriate regularization, a neural network may outperform a decision tree, so we cannot conclude that one algorithm is better than the other in general.\par
\bigskip
Even if we had a larger dataset, we need to tune hyperparameters. Without doing so, the decision trees and neural nets have not been trained properly.\par
\bigskip
In general, no algorithm will perform better than all other algorithms on all datasets (No free lunch theorem).\par
\bigskip
Finally, there may be other criteria for deciding the best learning algorithm. Even if our neural net has a lower accuracy than a decision tree (which is unlikely), the former does not require hand-coded features like facial action units. On the other hand, decision trees are often considered to be more interpretable i.e. a glass-box model rather than black-box model.

\section*{Question A2}

For the decision trees coursework, we need to train new decision trees and retrain existing decision trees, because the information gain from splitting on every attribute will change. Of course, we also need to map each emotion to their corresponding integer labels.\par
\bigskip
For the neural networks coursework, we only need to change the \texttt{num\_classes} parameter to the new number of classes we have. This will automatically increase the dimensions of our softmax output vector, so we don't need to encode emotions like in decision trees.\par
\bigskip
The entire neural net must be retrained since new emotions will affect the softmax probabilities of existing emotions. If we have normalized the training data by subtracting the mean image, then we need to recompute the mean image too.\par
\bigskip
For both learning algorithms, we need to retune all hyper-parameters by cross validation. For decision trees, this means tuning the maximum depth. For neural nets, this means adjusting the learning rate, network topology (number of neurons and layers) and regularization factor.

\section*{Question 6}

We have trained a convolutional neural network (CNN) with batch size = 32, number of epochs = 150 and the following architecture:
\begin{center}
\begin{tabular} { C{4cm} | L{11cm} }
    Data Argumentation & 1. Randomly rotate images 45 degrees\par
    2. Randomly (25\%) shift images horizontally and vertically\par
    3. Randomly flip images horizontal and vertically \\ \hline

    Convolution2D & 2 convolutional layers of 32 kernels with size of $3 \times 3$ using ReLU activation \\ \hline

    Optimization & $2 \times 2$ Maxpool Layer\par
    (and dropout with probability 0.25) \\ \hline

    Convolution2D & 2 convolutional layers of 64 kernels with size of $3 \times 3$ using ReLU activation \\ \hline

    Optimization & $2 \times 2$ Maxpool Layer with dropout probability = 0.25 \\ \hline

    Fully Connected Network & 512 neurons\par
    Activation: ReLU\par
    Dropout with probability 0.5\par
    Activation: Softmax \\
\end{tabular}
\end{center}
\bigskip

Confusion matrix based on test data:\par
\begin{center}
\begin{tabular} { C{2.7cm} | C{1.5cm} C{1.5cm} C{1.3cm} C{1.7cm} C{1.5cm} C{1.5cm} C{1.5cm} }
    \multirow{2}{*}{Actual Class} & \multicolumn{7}{c}{Predicted Class} \\
    & Anger & Disgust & Fear & Happiness & Sadness & Surprise & Neutral \\ \hline
    Anger     & 242 &  3 &  51 &  21 &  76 &   8 &  66 \\
    Disgust   &  23 & 11 &   6 &   2 &   9 &   2 &   3 \\
    Fear      &  70 &  0 & 143 &  18 & 140 &  41 &  84 \\
    Happiness &  41 &  0 &  14 & 714 &  27 &  19 &  80 \\
    Sadness   &  62 &  0 &  60 &  37 & 313 &   6 & 175 \\
    Surprise  &  13 &  0 &  59 &  19 &  18 & 293 &  13 \\
    Neutral   &  43 &  0 &  31 &  50 &  87 &   5 & 391
\end{tabular}
\end{center}
Summary statistics:
\begin{center}
\begin{table} [h!]
\begin{tabular} { C{2.7cm} | C{1.5cm} C{1.5cm} C{1.3cm} C{1.7cm} C{1.5cm} C{1.5cm} C{1.5cm} }
    & Anger & Disgust & Fear & Happiness & Sadness & Surprise & Neutral \\ \hline
    Precision   & 0.490 & 0.786 & 0.393 & 0.829 & 0.467 & 0.783 & 0.482 \\
    Recall      & 0.518 & 0.196 & 0.288 & 0.798 & 0.479 & 0.706 & 0.644 \\
    $F_1$ score & 0.504 & 0.314 & 0.333 & 0.813 & 0.473 & 0.743 & 0.551
\end{tabular}
\[ \text{Classification rate} = \frac{242 + 11 + \dotsm + 391}{3589} = 0.587 \]
\caption{Convolutional Neural Network} \label{CNN}
\end{table}
\end{center}

Comparing Table \ref{CNN} with Table \ref{simple}, it is clear that the precision, recall and $F_1$ scores of the CNN are better than the simple NN except that the recall of \textit{Fear} is worse for CNN. After using the CNN, the classification rate improves nearly 14\%.\par
\bigskip
This isn't surprising because in the CNN, we use data argumentation techniques, which generates transformed images to train the CNN. The learned weights will be more robust to various transformations and hence less likely to overfit.\par
\bigskip
Moreover, the convolution kernel helps to extract ``local'' properties of an image. After using a set of filters, the CNN will learn the most important pattern of images trained. In addition, the maxpool layers will downsample the outputs to reduce the size of representation, which will simplify the data and improve the accuracy. While, in NN, the pictures are read as input neurons, which contain many trivial pixels in data points, so the accuracy will be worse than the CNN.\par

\begin{figure} [h!]
    \centering
    \includegraphics[width=\textwidth] {cnn.png}
    \caption{Architecture of Convolutional Neural Network}
\end{figure}

\end{document}
